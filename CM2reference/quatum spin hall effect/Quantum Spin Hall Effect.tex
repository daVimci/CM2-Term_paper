\documentclass[prl,aps,amssymb,shownopacs,twocolumn]{revtex4}
\usepackage{amsmath}
\usepackage{amssymb}
\usepackage{amsthm}
\usepackage{amsfonts}
%\usepackage{algorithmic}
\usepackage{enumerate}
\usepackage{latexsym}
\usepackage[dvips]{graphicx}


\newcommand{\beq}{\begin{equation}}
	\newcommand{\eneq}{\end{equation}}

\input{epsf}

\begin{document}
	
	\tolerance 10000
	
	
	%\draft
	
	\title{Quantum Spin Hall Effect}
	
	\author {B. Andrei Bernevig and Shou-Cheng Zhang}
	
	\affiliation{  Department of Physics, Stanford University, Stanford,
		CA 94305}
	\begin{abstract}
		%\vspace*{-1.0truecm}
		\begin{center}
			
			\parbox{14cm}{The quantum Hall liquid is a novel state of matter
				with profound emergent properties such as fractional charge and
				statistics. Existence of the quantum Hall effect requires breaking
				of the time reversal symmetry caused by an external magnetic field.
				In this work, we predict a quantized spin Hall effect in the absence
				of any magnetic field, where the intrinsic spin Hall conductance is
				quantized in units of $2 \frac{e}{4\pi}$. The degenerate quantum
				Landau levels are created by the spin-orbit coupling in conventional
				semiconductors in the presence of a strain gradient. This new state
				of matter has many profound correlated properties described by a
				topological field theory.}
			
		\end{center}
	\end{abstract}
	%\pacs{73.43.-f,72.25.Dc,72.25.Hg,85.75.-d}
	
	
	\maketitle
	
	Recently, the intrinsic spin Hall effect has been theoretically
	predicted for semiconductors with spin-orbit coupled band
	structures\cite{murakami2003,sinova2004}. The spin Hall current is
	induced by the external electric field according to the equation
	\begin{equation}
		j_j^i = \sigma_s \epsilon_{ijk} E_k \label{spin_response}
	\end{equation}
	where $j_j^i$ is the spin current of the $i$-th component of the
	spin along the direction $j$, $E_k$ is the electric field and
	$\epsilon_{ijk}$ is the totally antisymmetric tensor in three
	dimensions. The spin Hall effect has recently been detected in two
	different experiments\cite{kato2004A,wunderlich2004}, and there is
	strong indication that at least one of them is in the intrinsic
	regime\cite{bernevig2004A}. Because both the electric field and
	the spin current are even under time reversal, the spin current
	could be dissipationless, and the value of $\sigma_s$ could be
	independent of the scattering rates. This is in sharp contrast
	with the extrinsic spin Hall effect, where the effect arises only
	from the Mott scattering from the impurity atoms\cite{mott1929}.
	
	The independence of the intrinsic spin Hall conductance $\sigma_s$
	on the impurity scattering rate naturally raises the question
	whether it can be quantized under certain conditions, similar to the
	quantized charge Hall effect. We start off our analysis with a
	question: Can we have Landau Level (LL) -like behavior \emph{in the
		absence} of a magnetic field \cite{haldane1988}? The quantum Landau
	levels arise physically from a {\it velocity dependent force},
	namely the Lorentz force, which contributes a term proportional to
	$\vec{A}\cdot \vec{p}$ in the Hamiltonian. Here $\vec{p}$ is the
	particle momentum and $\vec{A}$ is the vector potential, which in
	the symmetric gauge is given by $\vec{A} = \frac{B}{2}(y, -x,0)$. In
	this case, the velocity dependent term in the Hamiltonian is
	proportional to $B(xp_y-yp_x)$.
	
	In condensed matter systems, the only other ubiquitous velocity
	dependent force besides the Lorentz force is the spin-orbit coupling
	force, which contributes a term proportional to $(\vec{p} \times
	\vec{E})\cdot \vec{\sigma}$ in the Hamiltonian. Here $\vec{E}$ is
	the electric field, and $\vec{\sigma}$ is the Pauli spin matrix.
	Unlike the magnetic field, the presence of an electric field does
	not break the time reversal symmetry. If we consider the particle
	momentum confined in a two dimensional geometry, say the $xy$ plane,
	and the electric field direction confined in the $xy$ plane as well,
	only the $z$ component of the spin enters the Hamiltonian.
	Furthermore, if the electric field $\vec{E}$ is not constant but is
	proportional to the radial coordinate $\vec{r}$, as it would be, for
	example, in the interior of a uniformly charged cylinder $\vec{E}
	\sim E(x,y,0)$, then the spin-orbit coupling term in the Hamiltonian
	takes the form $E \sigma_z (xp_y-yp_x)$. We see that this system
	behaves in such a way as if particles with opposite spins experience
	the opposite ``effective" orbital magnetic fields, and a Landau
	level structure should appear for each spin orientations.
	
	However, such an electric field configuration is not easy to
	realize. Fortunately, the scenario previously described is
	realizable in zinc-blende semiconductors such as GaAs, where the
	shear strain gradients can play a similar role. Zinc-blende
	semiconductors have the point-group symmetry $T_d$ which is half
	of the cubic-symmetry group $O_h$, and does not contain inversion
	as one of its symmetries. Under the $T_d$ point group, the cubic
	harmonics $xyz$ transform like the identity, and off-diagonal
	symmetric tensors ($xy +yx$, etc.) transform in the same way as
	vectors on the other direction ($z$, etc), and represent basis
	functions for the $T_1$ representation of the group. Specifically,
	strain is a symmetric tensor $\epsilon_{ij} = \epsilon_{ji}$, and
	its off-diagonal (shear) components are, for the purpose of
	writing down a spin-orbit coupling Hamiltonian, equivalent to an
	electric field in the remaining direction:
	\begin{equation} \label{strainelectricfieldanalogy}
		\epsilon_{xy} \leftrightarrow E_z;\;\;\; \epsilon_{xz} \leftrightarrow
		E_y;\;\;\;\epsilon_{yz} \leftrightarrow E_x
	\end{equation}
	\noindent The Hamiltonian for the conduction band of bulk
	zinc-blende semiconductors under strain is hence the analogous to
	the spin-orbit coupling term $(\vec{v} \times \vec{E})\cdot
	\vec{\sigma}$. In addition, we have the usual kinetic $p^2$ term
	and a trace of the strain term $tr{\epsilon} = \epsilon_{xx} +
	\epsilon_{yy} +\epsilon_{zz}$, both of which transform as the
	identity under $T_d$:
	\begin{eqnarray}
		& H = \frac{p^2}{2m} + B tr{\epsilon} +\frac{1}{2}
		\frac{C_3}{\hbar}
		[ (\epsilon_{xy} p_y - \epsilon_{xz} p_z) \sigma_x + \nonumber \\
		& +(\epsilon_{zy} p_z - \epsilon_{xy} p_x) \sigma_y + (\epsilon_{zx}
		p_x - \epsilon_{yz} p_y) \sigma_z ]
	\end{eqnarray}
	\noindent For GaAs, the constant $\frac{C_3}{\hbar} = 8 \times
	10^5 m/s$ \cite{dyakonov1986}. This Hamiltonian is not new and was
	previously written down in Ref.
	\cite{pikus1984,howlett1977,khaetskii2001,bahder1990} but the
	analogy with the electric field and its derivation from a Lorentz
	force is suggestive enough to warrant repetition. There is yet
	another term allowed by group theory in the Hamiltonian
	\cite{bernevig2004}, but this is higher order in perturbation
	theory and hence not of primary importance.
	
	Let us now presume a strain configuration in which $\epsilon_{xy}
	= 0$ but $\epsilon_{xz}$ has a constant gradient along the $y$
	direction while $\epsilon_{yz}$ has a constant gradient along the
	$x$ direction. This case then mimics the situation of the electric
	field inside a uniformly charged cylinder discussed above, as
	$\epsilon_{xz} (\leftrightarrow E_y) =g y$ and $\epsilon_{yz}
	(\leftrightarrow E_x) = g x$, $g$ being the magnitude of the
	strain gradient. With this strain configuration and in a symmetric
	quantum well in the $xy$ plane, which we approximate as being
	parabolic, the above Hamiltonian becomes:
	\begin{equation}
		H= \frac{p_x^2 + p_y^2}{2m} + \frac{1}{2} \frac{C_3}{\hbar}g (y p_x
		- x  p_y)\sigma_z + D(x^2 + y^2)
	\end{equation}
	\noindent We first solve this Hamiltonian and come back to the
	experimental realization of the strain architecture in the later
	stages of the paper. We make the coordinate change $x \rightarrow
	(2mD)^{-1/4} x$, $y \rightarrow (2mD)^{-1/4} y$ and $R =
	\frac{1}{2} \frac{C_3}{\hbar} \sqrt{\frac{2m}{D}} g$. $R=2$ or
	$D=D_0\equiv\frac{2mgC_3^2}{16\hbar^2}$ is a special point, where
	the Hamiltonian can be written as a complete square, namely $H =
	\frac{1}{2m} (\vec{p} - e \vec{A} \sigma_z)^2$ with ${\vec{A} =
		\frac{m C_3 g}{2 \hbar e}(y, -x, 0)}$. At this point, our
	Hamiltonian is exactly equivalent to the usual Hamiltonian of a
	charged particle in an uniform magnetic field, where the two
	different spin directions experience the opposite directions of
	the ``effective" magnetic field. Any generic confining potential
	$V(x,y)$ can be written as $D_0(x^2 + y^2) + \Delta V(x,y)$, where
	the first term completes the square for the Hamiltonian, and the
	second term $\Delta V(x,y)=V(x,y)-D_0(x^2 + y^2)$ describes the
	additional static potential within the Landau levels. Since $[H,
	\sigma_z] =0$ we can use the spin on the $z$ direction to
	characterize the states. In the new coordinates, the Hamiltonian
	takes the form:
	\begin{eqnarray}
		& H = \left(%
		\begin{array}{cc}
			H_{\uparrow} & 0 \\
		\label{key}	0 & H_{\downarrow} \\
		\end{array}%
		\right) \nonumber \\ & H_{ \downarrow , \uparrow} =
		\sqrt{\frac{D}{2m}} [p_x^2 + p_y^2 + x^2+ y^2 \pm R(x p_y - y p_x)]
	\end{eqnarray}
	\noindent The $H_{ \downarrow, \uparrow}$ is the Hamiltonian for the
	down and up spin $\sigma_z$ respectively. Working in
	complex-coordinate formalism and choosing $z = x+ i y$ we obtain two
	sets of raising and lowering operators:
	\begin{eqnarray}
		& a = \partial_{z^\star} + \frac{z}{2}, \;\;\; a^\dagger = - \partial_z + \frac{z^\star}{2} \nonumber \\
		& b = \partial_{z} + \frac{z^\star}{2}, \;\;\; b^\dagger = -
		\partial_{z^\star} + \frac{z}{2}
	\end{eqnarray}
	\noindent in terms of which the Hamiltonian decouples:
	\begin{equation}
		H_{ \downarrow, \uparrow} =2 \sqrt{\frac{D}{2m}} \left[(1 \mp
		\frac{R}{2} ) a a^\dagger + (1 \pm \frac{R}{2}) b b^\dagger  + 1
		\right]
	\end{equation}
	\noindent The eigenstates of this system are harmonic oscillators
	$|m,n\rangle = (a^\dagger)^m (b^\dagger)^n |0,0\rangle$ of energy
	$E^{ \downarrow, \uparrow}_{m,n} = \frac{1}{2} \sqrt{\frac{D}{2m}}
	\left[(1 \mp \frac{R}{2} ) m + (1 \pm \frac{R}{2}) n + 1 \right]$.
	We shall focus on the case of $R=2$ where there is no additional
	static potential within the Landau level.
	
	For the spin up electron, the vicinity of $R \approx 2$ is
	characterized by the Hamiltonian $H_{\uparrow} = \frac{1}{2}
	\frac{C_3}{\hbar} g (2 a a^\dagger + 1) $ with the LLL wave function
	$\phi^\uparrow_n(z) =\frac{z^n}{\sqrt{\pi n ! } }\exp(\frac{-z
		z^\star}{2})$. $a$ is the operator moving between different Landau
	levels, while $b$ is the operator moving between different
	degenerate angular momentum states within the same LL: $L_z= b
	b^\dagger - a a^\dagger$, $L_z \phi^\uparrow_n(z) =  n
	\phi^\uparrow_n(z)$. The wave function, besides the confining
	factor, is holomorphic in $z$, as expected. These up spin electrons
	are the chiral, and their charge conductance is quantized in units
	of $e^2/h$.
	
	
	For the spin down electron, the situation is exactly the opposite.
	The vicinity of $R \approx 2$ is characterized by the Hamiltonian
	$H_{\downarrow} = \frac{1}{2} \frac{C_3}{\hbar} g (2 b b^\dagger +
	1) $ with the LLL wave function $\phi^\downarrow_m(z)
	=\frac{(z^\star)^m}{\sqrt{\pi m ! } }\exp(\frac{-z z^\star}{2})$.
	$b$ is the operator moving between different Landau levels, while
	$a$ is the operator between different degenerate angular momentum
	states within the same LL: $L_z= b b^\dagger - a a^\dagger$, $L_z
	\phi^\downarrow_m(z) = - m \phi^\downarrow_m(z)$. The wave function,
	besides the confining factor, is anti-holomorphic in $z$. These down
	spin electrons are anti-chiral, and their charge conductance is
	quantized in units of $-e^2/h$.
	

	
	The picture that now emerges is the following: our system is
	equivalent to a bilayer system; in one of the layers we have spin
	down electrons in the presence of a down-magnetic field whereas in
	the other layer we have spin up electrons in the presence of an
	up-magnetic field. These two layers are placed together. The spin up
	electrons have positive charge Hall conductance while the spin down
	electrons have negative charge Hall conductance. As such, the charge
	Hall conductance of the whole system vanishes. The time reversal
	symmetry reverses the directions of the ``effective" orbital
	magnetic fields, but interchanges the layers at the same time.
	However, the spin Hall conductance remains finite, as the chiral
	states are spin-up while the anti-chiral states are spin-down, as
	shown in Figure[\ref{edgecurrent}]. The spin Hall conductance is
	hence quantized in units of $2 \frac{e^2}{h}
	\frac{\hbar}{2e}=2\frac{e}{4\pi}$. Since an electron with charge $e$
	also carries spin $\hbar/2$, a factor of $\frac{\hbar}{2e}$ is used
	to convert charge conductance into the spin conductance.
	
	
	While it is hard to experimentally measure the spin Hall effect, and
	supposedly even harder to measure the quantum version of it, the
	measurement of the charge quantum Hall effect has become relatively
	common. In our system, however, the charge Hall conductance
	$\sigma_{xy}$ vanishes by symmetry. However, we can use our physical
	analogy of the two layer system placed together. In each of the
	layers we have a charge Quantum Hall effect at the same time (since
	the filling is equal in both layers), but with opposite Hall
	conductance. However, when on the plateau, the longitudinal
	conductance $\sigma_{xx}$ also vanishes ($\sigma_{xx} =0$)
	separately for the spin-up and spin-down electrons, and hence
	vanishes for the whole system. Of course, between plateaus it will
	have non-zero spikes (narrow regions). This is the easiest
	detectable feature of the new state, as the measurement is entirely
	electric. Other experiments on the new state could involve the
	injection of spin-polarized edge states, which would acquire
	different chirality depending on the initial spin direction.
	
	We now discuss the realization of a strain gradient of the
	specific form proposed in this paper. The strain tensor is related
	to the displacement of lattice atoms from their equilibrium
	position $u_i$ in the familiar way $\epsilon_{ij} = (\partial u_i
	/\partial x_j +
	\partial u_j /\partial x_i)/2$. Our strain configuration is
	$\epsilon_{xx} =\epsilon_{yy} = \epsilon_{zz} = \epsilon_{xy}=0$ as
	well as the strain gradients $\epsilon_{zx} = g y$ and
	$\epsilon_{yx} = g x$. Having the diagonal strain components
	non-zero will not change the physics as they add only a chemical
	potential term to the Hamiltonian. The above strain configuration
	corresponds to a displacement of atoms from their equilibrium
	positions of the form $\vec{u} = (0,0, 2 g x y)$. This can be
	possibly realized by pulverizing GaAs on a substrate in MBE at a
	rate which is a function of the position of the pulverizing beam on
	the substrate. The GaAs pulverization rate should vary as $xy \sim
	r^2 \sin(2\phi)$, where $r$ is the distance from one of the corners
	of the sample where the GaAs depositing was started. Conversely, we
	can keep the pulverizing beam fixed at some $r$ and rotate the
	sample with an angle-dependent angular velocity of the form $\sin(2
	\phi)$. We then move to the next incremental distance $r$, increase
	the beam rate as $r^2$ and again start rotating the substrate as the
	depositing procedure is underway.
	
	
	
	The strain architecture we have proposed to realize the Quantum Spin
	Hall effect is by no means unique. In the present case, we have
	re-created the so-called symmetric gauge in magnetic-field language,
	but, with different strain architectures, one can create the Landau
	gauge Hamiltonian and indeed many other gauges. The Landau gauge
	Hamiltonian is maybe the easiest to realize in an experimental
	situation, by growing the quantum well in the $[110]$ direction.
	This situation creates an off-diagonal strain $\epsilon_{xy}  =
	\frac{1}{4} S_{44} T$, and $\epsilon_{xz} = \epsilon_{yz} =0$ where
	$T$ is the lattice mismatch (or impurity concentration), $s_{44}$ is
	a material constant and $x,y,z$ are the cubic axes. The spin-orbit
	part of the Hamiltonian is now $\frac{C_3}{\hbar} \epsilon_{xy} (p_x
	\sigma_y - p_y \sigma_x)$. However, since the growth direction of
	the well is $[110]$ we must make a coordinate transformation to the
	$x',y',z'$ coordinates of the quantum well ($x', y'$ are the new
	coordinates in the plane of the well, whereas $z'$ is the growth
	coordinate, perpendicular to the well and identical to the $[110]$
	direction in cubic axes). The coordinate transformation reads: $x'
	=\frac{1}{\sqrt{2} }(x-y)$, $y' = -z$, $z' = \frac{1}{\sqrt{2}}
	(x+y)$, and the momentum along $z'$ is quantized. We now vary the
	impurity concentration $T$ (or vary the speed at which we deposit
	the layers) linearly on the $y'$ direction of the quantum well so
	that $\epsilon_{xy}= g y'$ where $g$ is strain gradient, linearly
	proportional to the gradient in $T$. In the new coordinates and for
	this strain geometry, the Hamiltonian reads:
	\begin{equation}
		H = \frac{p^2}{2m} + \frac{C_3}{\hbar} g y' p_{x'} \sigma_{z'} + D
		y^2
	\end{equation}
	\noindent where we have added a confining potential. At the
	suitable match between $D$ and $g$, this is the Landau-gauge
	Hamiltonian. One can also replace the soft-wall condition (the
	Harmonic potential) by hard-wall boundary condition.
	
	We now estimate the Landau Level gap and the strain gradient
	needed for such an effect, as well as the strength of the
	confining potential. In the case $R \approx 2$ the energy
	difference between Landau levels is $\Delta E_{Landau} = 2 \times
	\hbar \frac{1}{2} \frac{C_3}{\hbar} g = C_3 g$. For a gap of
	$1mK$, we hence need a strain gradient or $1 \%$ over $60 \mu m$.
	Such a strain gradient is easily realizable experimentally, but
	one would probably want to increase the gap to $10mK$ or more, for
	which a strain gradient of $1 \%$ over $6 \mu m$ or larger is
	desirable. Such strain gradients have been realized
	experimentally, however, not exactly in the configuration proposed
	here \cite{shen1996,shen1997}. The strength of the confining
	potential is in this case $D = 10^{-15} N/m$ which corresponds
	roughly to an electric field of $1 V/m$ for a sample of $60 \mu
	m$. In systems with higher spin-orbit coupling, $C_3$ would be
	larger, and the strain gradient field would create a larger gap
	between the Landau levels.
	
	We now turn to the question of the many-body wave function in the
	presence of interactions. For our system this is very suggestive, as
	the wave function incorporates both holomorphic and anti-holomorphic
	coordinates, by contrast to the pure holomorphic Laughlin states.
	Let the up-spin coordinates be $z_i$ while the down-spin coordinates
	be the $w_i$. $z_i$ enter in holomorphic form in the wave function
	whereas $w_i$ enter anti-holomorphically. While if the spin-up and
	spin-down electrons would lie in separate bi-layers the many-body
	wave function would be just $\prod_{i<j} (z_i -z_j)^m \prod_{k<l}
	(w^\star_k - w^\star_l)^m e^{-\frac{1}{2} (\sum_{i}z_i z_i^\star +
		\sum_k w_k w_k^\star)}$, where $m$ is an odd integer. Since the
	particles in our state reside in the same quantum well and may
	possibly experience the additional interaction between the different
	spin states, a more appropriate wave function is:
	\begin{eqnarray}
		& \psi(z_i, w_i) = \prod_{i<j} (z_i -z_j)^m \prod_{k<l} (w^\star_k
		- w^\star_l)^m \nonumber \\ & \prod_{r,s} (z_r -w^\star_s)^n
		e^{-\frac{1}{2} (\sum_{i} z_i z_i^\star + \sum_k w_k w_k^\star)}
	\end{eqnarray}
	\noindent The above wave function is symmetric upon the interchange
	$z \leftrightarrow w^\star$ reflecting the spin-$\uparrow$ chiral -
	spin-$\downarrow$ anti-chiral symmetry. This wave function is of
	course analogous to the Halperin's wave function of two different
	spin states \cite{halperin1983}. The key difference is that the two
	different spin states here experience the opposite directions of
	magnetic fields.
	
	Many profound topological properties of the quantum Hall effect
	are captured by the Chern-Simons-Landau-Ginzburg
	theory\cite{zhang1989}. While the usual spin orbit coupling for
	spin-$1/2$ systems is $T$-invariant but $P$-breaking, our
	spin-orbit coupling is also $P$-invariant due to the strain
	gradient. The low energy field theory of the spin Hall liquid is
	hence a double Chern-Simons theory with the action:
	\begin{equation}
		S = \frac{\nu}{4 \pi} \int \epsilon^{\mu \nu \rho} a_\mu
		\partial_\nu a_\rho - \frac{\nu}{4 \pi} \int \epsilon^{\mu \nu \rho}
		c_\mu \partial_\nu c_\rho
	\end{equation}
	\noindent where the $a_\mu$ and $c_\mu$ fields are associated with
	the left and right movers of our theory while $\nu$ is the filling
	factor. The fractional charge and statistics of the quasi-particle
	follow easily from this Chern-Simons action. Essentially, the two
	Chern-Simons terms have the same filling factor $\nu$ and hence
	the Hilbert space is not the tensor product of any two algebras,
	but of two identical ones. This is a mathematical statement of the
	fact that one can insert an up or down electron in the system with
	the same probability. Such special theories avoids the chiral
	anomaly \cite{freedman2003} and their Berry phases have been
	recently proposed as preliminary examples of topological quantum
	computation \cite{freedman2003}. It is refreshing to see that such
	abstract mathematical models can be realized in conventional
	semiconductors.
	
	A similar situation of Landau levels without magnetic field arises
	in rotating BECs where the mean field Hamiltonian is similar to
	either $H_\uparrow$ or $H_\downarrow$. In the limit of rapid
	rotation, the condensate expands and becomes effectively
	two-dimensional. The $L_z$ term is induced by the rotation vector
	$\Omega$ \cite{ho2001}. The LLL behavior is achieved when the
	rotation frequency reaches a specific value analogous to the case
	$R\approx 2$ in our Hamiltonian. In the BEC literature this is the
	so called mean-field quantum Hall limit and the ground state wave
	function is Laughlin-type. In contrast to our case, the theory is
	still $T$ breaking, the magnetic field is replaced by a rotation
	axial vector, and the lowest Landau level is chiral.
	
	In conclusion we predict a new state of matter where a quantum
	spin Hall liquid is formed in conventional semiconductors with
	spin-orbit coupling. The quantum Landau levels are caused by the
	gradient of strain field, rather than the magnetic field. The new
	quantum spin Hall liquid state shares many emergent properties
	similar to the charge quantum Hall effect, however, unlike the
	charge quantum Hall effect, our system does not violate time
	reversal symmetry.
	
	
	We wish to thank S. Kivelson, E. Fradkin, J. Zaanen, D. Santiago and
	C. Wu for useful discussions. B.A.B. acknowledges support from the
	Stanford Graduate Fellowship Program. This work is supported by the
	NSF under grant numbers DMR-0342832 and the US Department of Energy,
	Office of Basic Energy Sciences under contract DE-AC03-76SF00515.
	
	
	\begin{thebibliography}{19}
		\expandafter\ifx\csname
		natexlab\endcsname\relax\def\natexlab#1{#1}\fi
		\expandafter\ifx\csname bibnamefont\endcsname\relax
		\def\bibnamefont#1{#1}\fi
		\expandafter\ifx\csname bibfnamefont\endcsname\relax
		\def\bibfnamefont#1{#1}\fi
		\expandafter\ifx\csname citenamefont\endcsname\relax
		\def\citenamefont#1{#1}\fi
		\expandafter\ifx\csname url\endcsname\relax
		\def\url#1{\texttt{#1}}\fi
		\expandafter\ifx\csname urlprefix\endcsname\relax\def\urlprefix{URL
		}\fi \providecommand{\bibinfo}[2]{#2}
		\providecommand{\eprint}[2][]{\url{#2}}
		
		\bibitem[{\citenamefont{Murakami et~al.}(2003)\citenamefont{Murakami, Nagaosa,
				and Zhang}}]{murakami2003}
		\bibinfo{author}{\bibfnamefont{S.}~\bibnamefont{Murakami}},
		\bibinfo{author}{\bibfnamefont{N.}~\bibnamefont{Nagaosa}}, \bibnamefont{and}
		\bibinfo{author}{\bibfnamefont{S.}~\bibnamefont{Zhang}},
		\bibinfo{journal}{Science} \textbf{\bibinfo{volume}{301}},
		\bibinfo{pages}{1348} (\bibinfo{year}{2003}).
		
		\bibitem[{\citenamefont{\text{J. Sinova} \emph{et. al.}}(2004)}]{sinova2004}
		\bibinfo{author}{\bibnamefont{\text{J. Sinova} \emph{et. al.}}},
		\bibinfo{journal}{Phys. Rev. Lett.} \textbf{\bibinfo{volume}{92}},
		\bibinfo{pages}{126603} (\bibinfo{year}{2004}).
		
		\bibitem[{\citenamefont{\text{Y. Kato} \emph{et. al.}}()}]{kato2004A}
		\bibinfo{author}{\bibnamefont{\text{Y. Kato} \emph{et. al.}}},
		\bibinfo{howpublished}{Science, 11 Nov 2004 (10.1126/science.1105514)}.
		
		\bibitem[{\citenamefont{\text{J. Wunderlich }
				\emph{et.al.}}()}]{wunderlich2004}
		\bibinfo{author}{\bibnamefont{\text{J. Wunderlich } \emph{et.al.}}},
		\bibinfo{howpublished}{cond-mat/0410295}.
		
		\bibitem[{\citenamefont{Bernevig and Zhang}({\natexlab{a}})}]{bernevig2004A}
		\bibinfo{author}{\bibfnamefont{B.}~\bibnamefont{Bernevig}} \bibnamefont{and}
		\bibinfo{author}{\bibfnamefont{S.}~\bibnamefont{Zhang}},
		\bibinfo{howpublished}{cond-mat/0411457}.
		
		\bibitem[{\citenamefont{Mott}(1929)}]{mott1929}
		\bibinfo{author}{\bibfnamefont{N.}~\bibnamefont{Mott}}, \bibinfo{journal}{Proc.
			Roy. Soc.} \textbf{\bibinfo{volume}{124}}, \bibinfo{pages}{425}
		(\bibinfo{year}{1929}).
		
		\bibitem[{hal()}]{haldane1988}
		\bibinfo{howpublished}{A related question was posed by F.D.M. Haldane (Phys.
			Rev. Lett.{ \bf{61}}, 2015 (1988)) on whether the Quantum Hall effect
			necessarily requires an external magnetic field or can arise as a consequence
			of magnetic ordering in two-dimensional systems. The state proposed breaks
			time-reversal due to a local magnetic flux density of zero total flux through
			the unit cell.}
		
		\bibitem[{\citenamefont{\text{M. I. Dyakonov} \emph{et.
					al.}}(1986)}]{dyakonov1986}
		\bibinfo{author}{\bibnamefont{\text{M. I. Dyakonov} \emph{et. al.}}},
		\bibinfo{journal}{Sov. Phys. JETP} \textbf{\bibinfo{volume}{63}},
		\bibinfo{pages}{655} (\bibinfo{year}{1986}).
		
		\bibitem[{\citenamefont{J.E.Pikus and Titkov}()}]{pikus1984}
		\bibinfo{author}{\bibnamefont{J.E.Pikus}} \bibnamefont{and}
		\bibinfo{author}{\bibfnamefont{A.}~\bibnamefont{Titkov}},
		\bibinfo{howpublished}{Optical Orientation (North Holland, Amsterdam, 1984,
			p. 73)}.
		
		\bibitem[{\citenamefont{W.Howlett and Zukotynski}(1977)}]{howlett1977}
		\bibinfo{author}{\bibnamefont{W.Howlett}} \bibnamefont{and}
		\bibinfo{author}{\bibfnamefont{S.}~\bibnamefont{Zukotynski}},
		\bibinfo{journal}{Phys. Rev. B} \textbf{\bibinfo{volume}{16}},
		\bibinfo{pages}{3688} (\bibinfo{year}{1977}).
		
		\bibitem[{\citenamefont{Khaetskii and Nazarov}(2001)}]{khaetskii2001}
		\bibinfo{author}{\bibfnamefont{A.}~\bibnamefont{Khaetskii}} \bibnamefont{and}
		\bibinfo{author}{\bibfnamefont{Y.}~\bibnamefont{Nazarov}},
		\bibinfo{journal}{Phys. Rev. B} \textbf{\bibinfo{volume}{64}},
		\bibinfo{pages}{125316} (\bibinfo{year}{2001}).
		
		\bibitem[{\citenamefont{T.B.Bahder}(1990)}]{bahder1990}
		\bibinfo{author}{\bibnamefont{T.B.Bahder}}, \bibinfo{journal}{Phys. Rev. B}
		\textbf{\bibinfo{volume}{41}}, \bibinfo{pages}{11992} (\bibinfo{year}{1990}).
		
		\bibitem[{\citenamefont{Bernevig and Zhang}({\natexlab{b}})}]{bernevig2004}
		\bibinfo{author}{\bibfnamefont{B.}~\bibnamefont{Bernevig}} \bibnamefont{and}
		\bibinfo{author}{\bibfnamefont{S.}~\bibnamefont{Zhang}},
		\bibinfo{howpublished}{cond-mat/0408442, accepted in Phys. Rev. B.}
		
		\bibitem[{\citenamefont{\text{Q. Shen} \emph{et. al.}}(1996)}]{shen1996}
		\bibinfo{author}{\bibnamefont{\text{Q. Shen} \emph{et. al.}}},
		\bibinfo{journal}{Phys. Rev. B} \textbf{\bibinfo{volume}{54}},
		\bibinfo{pages}{16381} (\bibinfo{year}{1996}).
		
		\bibitem[{\citenamefont{Shen and Kycia}(1997)}]{shen1997}
		\bibinfo{author}{\bibfnamefont{Q.}~\bibnamefont{Shen}} \bibnamefont{and}
		\bibinfo{author}{\bibfnamefont{S.}~\bibnamefont{Kycia}},
		\bibinfo{journal}{Phys. Rev. B} \textbf{\bibinfo{volume}{55}},
		\bibinfo{pages}{15791} (\bibinfo{year}{1997}).
		
		\bibitem[{\citenamefont{Halperin}(1983)}]{halperin1983}
		\bibinfo{author}{\bibfnamefont{B.}~\bibnamefont{Halperin}},
		\bibinfo{journal}{Helv. Phys. Acta} \textbf{\bibinfo{volume}{56}},
		\bibinfo{pages}{75} (\bibinfo{year}{1983}).
		
		\bibitem[{\citenamefont{Zhang et~al.}(1989)\citenamefont{Zhang, Hansson, and
				Kivelson}}]{zhang1989}
		\bibinfo{author}{\bibfnamefont{S.~C.} \bibnamefont{Zhang}},
		\bibinfo{author}{\bibfnamefont{H.}~\bibnamefont{Hansson}}, \bibnamefont{and}
		\bibinfo{author}{\bibfnamefont{S.}~\bibnamefont{Kivelson}},
		\bibinfo{journal}{Phys. Rev. Lett.} \textbf{\bibinfo{volume}{62}},
		\bibinfo{pages}{82} (\bibinfo{year}{1989}).
		
		\bibitem[{\citenamefont{\text{M. Freedman} \emph{et. al.}}()}]{freedman2003}
		\bibinfo{author}{\bibnamefont{\text{M. Freedman} \emph{et. al.}}},
		\bibinfo{howpublished}{cond-mat/0307511}.
		
		\bibitem[{\citenamefont{Ho}(2001)}]{ho2001}
		\bibinfo{author}{\bibfnamefont{T.}~\bibnamefont{Ho}}, \bibinfo{journal}{Phys.
			Rev. Lett.} \textbf{\bibinfo{volume}{87}}, \bibinfo{pages}{060403}
		(\bibinfo{year}{2001}).
		
	\end{thebibliography}
	
	
	
	
	
\end{document}
